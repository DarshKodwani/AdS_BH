\documentclass[aps,showpacs,onecolumn,floats,prd,superscriptaddress,nofootinbib]{revtex4-1} 
\usepackage{graphicx,amsmath,amssymb,amstext}
\usepackage{amssymb,amsbsy,amsfonts,amsthm,color}
\usepackage{epsfig}
%\usepackage{showkeys}
\usepackage{graphicx}
\usepackage{subfigure}
\usepackage{sidecap}
\usepackage{floatrow}
\graphicspath{{Figures/}}

\begin{document}

\title{Notes for AdS }

\author{Darsh Kodwani}
\email{dkodwani@physics.utoronto.ca}
\affiliation{Canadian Institute of Theoretical Astrophysics, 60 St George St, Toronto, ON M5S 3H8, Canada.}
\affiliation{University of Toronto, Department of Physics, 60 St George St, Toronto, ON M5S 3H8, Canada.}

\author{Ue-Li Pen}
\email{pen@cita.utoronto.ca}
\affiliation{Canadian Institute of Theoretical Astrophysics, 60 St George St, Toronto, ON M5S 3H8, Canada.}
\affiliation{Canadian Institute for Advanced Research, CIFAR program in Gravitation and Cosmology.}
\affiliation{Dunlap Institute for Astronomy \& Astrophysics, University of Toronto, AB 120-50 St. George Street, Toronto, ON M5S 3H4, Canada.}
\affiliation{Perimeter Institute of Theoretical Physics, 31 Caroline Street North, Waterloo, ON N2L 2Y5, Canada.}

\author{I-Sheng Yang}

\email{isheng.yang@gmail.com}
\affiliation{Canadian Institute of Theoretical Astrophysics, 60 St George St, Toronto, ON M5S 3H8, Canada.}
\affiliation{Perimeter Institute of Theoretical Physics, 31 Caroline Street North, Waterloo, ON N2L 2Y5, Canada.}

\begin{abstract}

\end{abstract}

\maketitle


\section{Introduction and Summary}

Choose the metric

\begin{equation}
	ds^2 = - \left( 1 - b^2r^2 \right) dt^2 + \left( 1 - b^2r^2 \right)^{-1} dr^2 + r^2 d \Omega_2^2
\end{equation}

where $b^2 = \frac{\Lambda}{3}$. For a black hole of mass $M$ we have

\begin{equation}
	ds^2 = - \left( 1 - \frac{2M}{r} - b^2 r^2 \right) dt^2 + \left( 1 - \frac{2M}{r} - b^2 r^2 \right)^{-1} dr^2 + r^2 d \Omega^2_2
\end{equation}

we know

\begin{equation}
	\frac{\Delta T}{T} = \sqrt{\frac{g_{00}(b, r_2) g_{00}(b, \Delta M, r_0)}{g_{00}(b, r_1) g_{00}(b, \Delta M, r_1)}} -1
\end{equation}

where

\begin{eqnarray}
	g_{00}(\Lambda, r) & = & - \left( 1 - b^2 r^2 \right)	\nonumber	\\
	g_{00}(\Lambda, \Delta M, r) & = & - \left( 1 - \frac{2 \Delta M}{r} - b^2 r^2 \right)
\end{eqnarray}

The induced metrics we will use are

\begin{equation}
	ds^2_I = g_{\alpha \beta} dx^\alpha dx^\beta  =-( 1 - b^2 r^2) dt^2 + (1 - b^2 r^2 ) dr^2 + r^2 d \Omega^2_2 
\end{equation}

\begin{equation}
	ds^2_{II} = g_{\alpha \beta} dx^\alpha dx^\beta  =- ( 1 - b^2 r^2) dt^2 + (1 - b^2 r^2 ) dr^2 + r^2 d \Omega^2_2 
\end{equation}

The induced metrics are

\begin{equation}
	ds^2_{3(I)} = h_{ab}^{(I)} dy^a dy^b = h_{00}^{(I)} d \hat{t}^2 + h_{11}^{(I)} d \hat{\theta}^2 + h^{(I)}_{22} d \hat{\phi}^2
\end{equation}

\begin{equation}
	ds^2_{3(II)} = h_{ab}^{(II)} dy'^a dy'^b = h_{00}^{(II)} d \hat{t'}^2 + h_{11}^{(II)} d \hat{\theta '}^2 + h^{(II)}_{22} d \hat{\phi '}^2
\end{equation}

matching the induced metrics at $r_1$ gives

\begin{equation}
	dt = \frac{g_{00}^{(II)}(r_1)}{g_{00}^{(I)}(r_1)} dt'
\end{equation}

We will need the stress energy of the shell and use the IJC for that. Lets look at the extrinsic curvature components first

\begin{equation}
	K^{(I)}_{ab} = \frac{1}{2} g^{(I)}_{\alpha \beta} e^\alpha_a e^\beta_b
\end{equation}

where 

\begin{equation}
	e^\alpha_a = \frac{d x^\alpha}{dy^a}
\end{equation}

which we can simplify 

\begin{equation}
	K_{tt}^{(I)} = \frac{1}{2} (g_{tt}^{(I)})^{-\frac{1}{2}}  \partial_r g_{tt}^{(I)}
\end{equation}

similarly the other components of the extrinsic curvature are

\begin{equation}
	K_{\theta \theta}^{(I)}= \frac{1}{2} n^r \partial_r g^{(I)}_{\theta \theta} = \frac{1}{2} (g_{tt}^{(I)})^{-\frac{1}{2}} \partial_r g_{\theta \theta}^{(I)} = K_{\phi \phi}^{(I)}
\end{equation}

\begin{equation}
	K_{tt}^{(II)} = \frac{1}{2} (g_{tt}^{(II)})^{-\frac{1}{2}}  \partial_r g_{tt}^{(II)}
\end{equation}

\begin{equation}
	K_{\theta \theta}^{(II)}= \frac{1}{2} n^r \partial_r g^{(II)}_{\theta \theta} = \frac{1}{2} (g_{tt}^{(II)})^{-\frac{1}{2}} \partial_r g_{\theta \theta}^{(II)} = K_{\phi \phi}^{(II)}
\end{equation}

The surface stress energy is

\begin{equation}
	S_{00} = \sigma = \frac{1}{4 \pi} ((K^\theta_\theta)^{(I)} - (K^\theta_{\theta})^{(II)})
\end{equation}

\begin{equation}
	\sigma = \frac{1}{4 \pi r_1^2} \left( \left( 1 - b^2r_1^2 - \frac{2 \Delta M}{r_1} \right)^{-\frac{1}{2}} - \left( 1- b^2 r_1^2 \right)^{-\frac{1}{2}} \right)
\end{equation}




























\bibliography{all_active}


\end{document}
